\documentclass[a4paper,10pt]{article}
\usepackage[utf8x]{inputenc}
\usepackage{amsmath}

\newcommand{\px}{\partial_x}
\newcommand{\py}{\partial_y}

%opening
\title{myQG}
\author{Nils Br\"uggemann}

\begin{document}

\maketitle

\section{Model equations}
asdf 
The diagnostic equation for quasi-geostrophic potential vorticity of layer $k$ $q_k$ reads as follows:
\begin{align}
 \partial_t q_k + u_k \partial_x q_k + v_k \partial_y q_k = F_k + D_k 
\end{align}
The velocity components $u_k$ and $v_k$ are derived by:
\begin{align}
 u_k &= - \py \psi_k,
&
 v_k &=   \px \psi_k.
\end{align}
The streamfunction $\psi_k$ can be obtained by inverting the following equation:
\begin{align}
 q_k = \beta y + \nabla^2 \psi_k + \frac{f_0}{H_k} \left( \frac{\psi_{k-1} - \psi_k}{g'_k} - \frac{\psi_k-\psi_{k+1}}{g'_{k+1}} \right)
\end{align}
where $H_k$ denotes a constant mean layer width, and $g'_{k}$ the reduced gravity that can be calculated as follows:
\begin{align}
 g'_k = \frac{g'_1}{\rho_1} ( \rho_k - \rho_{k-1} )
\end{align}
with $g'_1 = g$ and $\rho_k$ the densitiy of layer $k$.
The layer width $h_k$ of each layer can be calculated by
\begin{align}
 h_k = H_k + \eta_k
\end{align}
with $\eta_k$ given by:
\begin{align}
 \eta_k
\end{align}






\end{document}
